% ======================= Pre-Amble =========================
      
%Format
\documentclass[11pt, oneside]{article}   	% use "amsart" instead of "article" for AMSLaTeX format 
                     						%imports package {article} and specify option(s) [11pt, oneside]
\usepackage{geometry}                		% See geometry.pdf to learn the layout options. There are lots. 
    \geometry{letterpaper}                   		% ... or a4paper or a5paper or ... 
    %\geometry{landscape}                		% Activate for rotated page geometry

\usepackage[parfill]{parskip}    		        % Activate to begin paragraphs with an empty line rather than an indent

    %Colours
    \usepackage{graphicx, subcaption}
    \usepackage[usenames, dvipsnames]{color}     % font colour:    \textcolor{<colour>}{text}
          									%highlight text:  \colorbox{<color>}{text}
    \usepackage{soul}						%highlight text: \hl{}     %only  yellow								
    									%list of colours: https://www.sharelatex.com/learn/Using_colours_in_LaTeX
    									
    %Bullets
    \usepackage{enumerate}     %specify type of enumeration: \being{enumerate}[<type of enumeration>]
    
    %Footnote Spacing
    \setlength{\footnotesep}{0.4cm}                  %specify spacing b/w footnotes
    \setlength{\skip\footins}{0.6cm}                    % space b/w footnotes and textbody


%Mattematics
    %American Mathematics Society packages
    \usepackage{amsmath}	   %math
    \usepackage{amssymb}       %symbols
    \usepackage{amsthm}          %theorems

    %QED
    \newcommand*{\QEDA}{\hfill\ensuremath{\blacksquare}}         %make qed filled square:    \QEDA
    \newcommand*{\QEDB}{\hfill\ensuremath{\square}}               %make qed empty square: \QEDB 
    
    \renewcommand\qedsymbol{\ensuremath{\blacksquare}}		%Proof environment


%Figures
\usepackage{caption}
\captionsetup[figure]{labelfont=bf}    %make figure labels boldface
\captionsetup[table]{labelfont=bf}     %make table labels boldface

\usepackage[hidelinks]{hyperref}                % Allows for clickable references

    %Tables
    \usepackage[none]{hyphenat}                    % Stops breaking-up words in a table (i.e. no hyphens)                                                             
    
    \usepackage{array}   
    \newcolumntype{x}[1]{>{\centering\let\newline\\\arraybackslash\hspace{0pt}}p{#1}}       %center fixed column width: x{<len>}                      
    \newcolumntype{$}{>{\global\let\currentrowstyle\relax}}                                                   % let us apply things (e.g. bold/italicize) to entire row            
    \newcolumntype{^}{>{\currentrowstyle}}
    \newcommand{\rowstyle}[1]{\gdef\currentrowstyle{#1} #1\ignorespaces}
    
    %Images
    \graphicspath{ {images/} }                          %directory that your images are located in within your current directory
    
    %Diagrams
    \usepackage[latin1]{inputenc}
    \usepackage{tikz}
    \usepackage{tkz-berge}
    \usetikzlibrary{shapes,arrows}


%Bibliography
\usepackage[numbers,sort&compress]{natbib}   %for multiple references: sorts  (i.e. [1,2] NOT [2, 1] )
                                           				  %                                     compresses (i.e. [1-3] )
\usepackage[nottoc]{tocbibind}                            %add bibliography to table of contents


%Miscellaneous
\usepackage{dirtytalk}    %quotations: use \say  


%================== Header & Footer =========================
\usepackage{fancyhdr}
\usepackage{lastpage}      %ensures you can reference LastPage (i.e. Page 2 of 10)

\renewcommand{\headrulewidth}{0.4pt}		%Decorative Header line: thickness={0.4pt}
\renewcommand{\footrulewidth}{0.4pt}		%Decorative Footer line: thickness={0.4pt}

\setlength{\headheight}{13.6pt} 		%space b/w top of page & header
\setlength{\headsep}{0.3in}		%space b/w page header and body

%Make Header & Footer    
\pagestyle{fancy}
    \lhead{Stephanie Knill} 		% controls the left corner of the header
    \chead{} 					% controls the center of the header
    \rhead{} 					% controls the right corner of the header
    \lfoot{} 					% controls the left corner of the footer
    \cfoot{Page~\thepage\ of \pageref{LastPage}} 				% controls the center of the footer
    												%Page~\thepage\  if just want Page x
    \rfoot{}			 		% controls the right corner of the footer

% =============================== Document ===================================
\begin{document}

% Title Page
\title{MATH 302 --- Assignment 7 \\
\line(1,0){360} \\              %(slope x, y){length of line}
}
\author{
Stephanie Knill \\
54882113 \\
Due: March 9, 2016}

\date{}                   % Activate:  display a given date (e.g. {August 4} ) or no date (empty {} )
                                    %No activate: display current date
\maketitle

%\thispagestyle{empty}                   %Remove header from this (first) page. Change empty -> plain to keep numbering
%								-> Doesn't matter in this case (b/c title page)
%\cleardoublepage


% ================= Questions ================

\section*{Question 1: Section 5.2 \#2}

Let us choose a number $U$ from the unit interval [0,1] with uniform distribution. Then we can find the cumulative distribution and density for the random variables:

\begin{enumerate}[(a)]
	\item $Y=U+2$
		
	Here, range[Y] = $[1/1, 1/2] = [1/2, 1]$ So the cumulative distribution function in the range[Y] is
	\begin{align*}
		F_Y(y) & = P(Y \leq y) \\
		& = P\Big(\frac{1}{U+1} \leq y\Big) \\
		& = P(1 \leq y(U+1)) \\
		& = P(U+1 \geq 1/y) \\
		& = P(U \geq 1/y -1) \\
		& = 1 - P(U \leq 1/y-1) \\
		& = 1 - (1/y-1) \\
		& = 2 - 1/y
	\end{align*}
	Thus the cumulative distribution function is given by
	\begin{align*}
		F_Y(y) = \begin{cases}
					0, \qquad & x < 1/2 \\
					2-1/y, & x \in [1/2, 1]\\
					0, &  x >1
				\end{cases}
	\end{align*}
	and the probability density funciton
	\begin{align*}
		f_Y(y) = \begin{cases}
					1/y^2, \qquad & x \in [1/2, 1] \\
					0, & x \notin [1/2, 1]
				\end{cases}
	\end{align*}
	
	\item $Y=ln(U+1)$
		
	Here, range[Y] = $[ln(1), ln(2)] = [0, ln(2)]$ So the cumulative distribution function in the range[Y] is
	\begin{align*}
		F_Y(y) & = P(Y \leq y) \\
		& = P(ln(U+1) \leq y) \\
		& = P(e^{ln(U+1)} \leq e^y) \\
		& = P(U+1 \geq e^y) \\
		& = P(U \geq e^y-1) \\
		& = e^y-1
	\end{align*}
	Thus the cumulative distribution function is given by
	\begin{align*}
		F_Y(y) = \begin{cases}
					0, \qquad & x < 0 \\
					e^y-1, & x \in [0, ln(2)]\\
					0, &  x >ln(2)
				\end{cases}
	\end{align*}
	and the probability density funciton
	\begin{align*}
		f_Y(y) = \begin{cases}
					1/y^2, \qquad & x \in [1/2, 1] \\
					0, & x \notin [1/2, 1]
				\end{cases}
	\end{align*}
\end{enumerate}

\section*{Question 2: Section 5.2 \#14}

A point $P$ in the unit square has coordinates $X$ and $Y$ chosen at random in the interval [0,1]. Let $D$ be the distance from $P$ to the nearest edge of the square and $E$ the distance to the nearest corner. Then we can compute the following probabilities:

\begin{enumerate}[(a)]
	\item $D <1/4$
		
	\begin{align*}
		P(D < 1/4) & = 1 -P(D \geq 1/4) \\
		& = 1 - \frac{\text{Area($E$)}}{\text{Area(unit square)}} \\
	\end{align*}
	Here, the $E$ is the event $\{D \geq 1/4\}$. This is the square with edge length
	$$1-(1/4 \cdot 2) = 1/2$$
	substituting this in
	\begin{align*}
		P(D < 1/4) & = 1 - \frac{1/2 \cdot 1/2}{1 \cdot 1} \\
		& = 3/4
	\end{align*}
	\item $E <1/4$
		
	\begin{align*}
		P(E < 1/4) & = \frac{\text{Area($E$)}}{\text{Area(unit square)}}
	\end{align*}
	Here, the $F$ is the event $\{E \geq 1/4\}$. This is 4 quarter circles with radius $r=1/4$:
	$$\text{Area($E$)} = 4 \cdot \frac{\pi r^2}{4} = \pi \cdot (1/4)^2 = \frac{\pi}{16}$$
	substituting this in
	\begin{align*}
		P(D < 1/4) & = \frac{\frac{\pi}{16}}{1 \cdot 1} \\
		& = \frac{\pi}{16}
	\end{align*}
		
\end{enumerate}


\section*{Question 3: Section 5.2 \#16}

Let $X$ be a random variable with density function
\begin{align*}
		f_X(x) = \begin{cases}
					cx(1-x), \qquad & 0 < x< 1 \\
					0, & \text{otherwise}
				\end{cases}
	\end{align*}
\begin{enumerate}[(a)]
	\item To find the value of $c$, let us use the property that	
	\begin{align*}
		\int_{0}^1 f_X(x) dx = \int_0^1 cx(1-x) dx & = 1 \\
		c \dot \int_0^1 x-x^2 dx & = 1 \\
		1/6 \cdot c = 1 \\
		c = 6
	\end{align*}
	\item The cumulative density function is given by
	\begin{align*}
		F_X(x) & = \int_{-\infty}^x f(t) dt \\
		& = \int_{-\infty}^0 6t(1-t) dt \\
		& = 6 \Big[\frac{x^2}{2}-\frac{x^3}{3}\Big] \\
		& =3x^2-2x^3 \\
		& = x^2(3-2x)
	\end{align*}
	\item The probability P($X < 1/4$) can be computed as
	\begin{align*}
		P(X \leq x) & = P(X \in [0, 1/4]) \\
		& = F(1/4)\\
		& = (1/4)^2(3-2(1/4)) \\
		& =\frac{5}{32}
	\end{align*}
\end{enumerate}


\section*{Question 4: Section 5.2 \#17}

Let $X$ be a random variable with cumulative distribution function
\begin{align*}
		F_X(x) = \begin{cases}
					0, \qquad & x<0 \\
					sin^2(\pi x/2), & 0 \leq x \leq 1 \\
					1, & 1 < x
				\end{cases}
\end{align*}
\begin{enumerate}[(a)]
	\item The density function is given by
	\begin{align*}
		f_X(x) & = F'_X(x) \\
		& = \begin{cases}
					\pi \cdot sin(\pi x/2) \cdot cos(\pi x/2), \qquad & x \in [0,1] \\
					0, & x \notin [0,1]
				\end{cases}
	\end{align*}
	\item The probability P($X < 1/4$) can be computed as
	\begin{align*}
		P(X \leq x) & = P(X \in [0, 1/4]) \\
		& = F(1/4)\\
		& = sin^2(\pi/2 \cdot 1/4) \\
		& = \frac{1}{4}(2-\sqrt{2}) \\
		& \approx 0.146
	\end{align*}
\end{enumerate}


\section*{Question 5: Section 5.2 \#18}

Let $X$ be a random variable with cumulative distribution function $F_X(x)$, and let $Y=X+b$, $Z=aX$, $W=aX+b$.

In the case where $a >0$, we have that
\begin{align*}
	F_Y(y) & = P(Y \leq y) \\
	& = P(X+b \leq y) \\
	& = P(X \leq y-b) \\
	& = F_X(y-b)
\end{align*}
\begin{align*}
	F_Z(z) & = P(Z \leq z) \\
	& = P(aX \leq z) \\
	& = P(X \leq \frac{z}{a}) \\
	& = F_X\Big(\frac{z}{a}\Big)
\end{align*}
\begin{align*}
	F_W(w) & = P(W \leq w) \\
	& = P(aX+b \leq w) \\
	& = P(X \leq \frac{w-b}{a}) \\
	& = F_X\Big(\frac{w-b}{a}\Big)
\end{align*}



\section*{Question 6: Section 5.2 \#19}

Let $X$ be a random variable with cumulative distribution function $F_X(x)$, and let $Y=X+b$, $Z=aX$, $W=aX+b$.

In the case where $a >0$, we have that
\begin{align*}
	f_Y(y) & =  \int F_Y(y) \, dy \\
	& = \int F_X(y-b) \, dx\\
	& = f_X(y-b)
\end{align*}
\begin{align*}
	f_Z(z) & = \int F_Y(z) \, dz \\
	& = \int F_X\Big(\frac{z}{a}\Big) \, \frac{1}{a}\cdot dx\\
	& = \frac{1}{a}\cdot f_X\Big(\frac{z}{a}\Big)
\end{align*}
\begin{align*}
	f_W(w) & = \int F_W(w) \, dw \\
	& = \int F_X\Big(\frac{w-b}{a}\Big) \, \frac{1}{a}\cdot dx\\
	& = \frac{1}{a}\cdot f_X\Big(\frac{w-b}{a}\Big)
\end{align*}


\section*{Question 7: Section 5.2 \#19}

Let $X$ be a random variable uniformly distributed over $[c,d]$and $Y=aX+b$. Let us find $a,b$ such that $Y$ is uniformly distributed over [0.1].

Since range[$X] = [c, d]$, then range[$Y] = [ac+b, ad+b] =[0,1]$.
Thus we have a system of equatinos
\begin{align}
	ac+b & = 0 \\
	ad+b & = 1
\end{align}
Subtracting equation (2) from equation (1)
\begin{align*}
	ac+b - (ad+b) & = 0 -1 \\
	a(c-d) & = -1 \\
	a & = \frac{1}{d-c}
\end{align*}
Solving now for $b$
\begin{align*}
	b & = -ac \\
	& = \frac{c}{d-c} \\
	& = \frac{c}{d} - 1
\end{align*}


\section*{Question 8: Section 5.2 \#21}

I have no ducking clue. Pls don't mark.




\end{document} 