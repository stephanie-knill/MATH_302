% ======================= Pre-Amble =========================
      
%Format
\documentclass[11pt, oneside]{article}   	% use "amsart" instead of "article" for AMSLaTeX format 
                     						%imports package {article} and specify option(s) [11pt, oneside]
\usepackage{geometry}                		% See geometry.pdf to learn the layout options. There are lots. 
    \geometry{letterpaper}                   		% ... or a4paper or a5paper or ... 
    %\geometry{landscape}                		% Activate for rotated page geometry

\usepackage[parfill]{parskip}    		        % Activate to begin paragraphs with an empty line rather than an indent

    %Colours
    \usepackage{graphicx, subcaption}
    \usepackage[usenames, dvipsnames]{color}     % font colour:    \textcolor{<colour>}{text}
          									%highlight text:  \colorbox{<color>}{text}
    \usepackage{soul}						%highlight text: \hl{}     %only  yellow								
    									%list of colours: https://www.sharelatex.com/learn/Using_colours_in_LaTeX
    									
    %Bullets
    \usepackage{enumerate}     %specify type of enumeration: \being{enumerate}[<type of enumeration>]
    
    %Footnote Spacing
    \setlength{\footnotesep}{0.4cm}                  %specify spacing b/w footnotes
    \setlength{\skip\footins}{0.6cm}                    % space b/w footnotes and textbody


%Mattematics
    %American Mathematics Society packages
    \usepackage{amsmath}	   %math
    \usepackage{amssymb}       %symbols
    \usepackage{amsthm}          %theorems

    %QED
    \newcommand*{\QEDA}{\hfill\ensuremath{\blacksquare}}         %make qed filled square:    \QEDA
    \newcommand*{\QEDB}{\hfill\ensuremath{\square}}               %make qed empty square: \QEDB 
    
    \renewcommand\qedsymbol{\ensuremath{\blacksquare}}		%Proof environment


%Figures
\usepackage{caption}
\captionsetup[figure]{labelfont=bf}    %make figure labels boldface
\captionsetup[table]{labelfont=bf}     %make table labels boldface

\usepackage[hidelinks]{hyperref}                % Allows for clickable references

    %Tables
    \usepackage[none]{hyphenat}                    % Stops breaking-up words in a table (i.e. no hyphens)                                                             
    
    \usepackage{array}   
    \newcolumntype{x}[1]{>{\centering\let\newline\\\arraybackslash\hspace{0pt}}p{#1}}       %center fixed column width: x{<len>}                      
    \newcolumntype{$}{>{\global\let\currentrowstyle\relax}}                                                   % let us apply things (e.g. bold/italicize) to entire row            
    \newcolumntype{^}{>{\currentrowstyle}}
    \newcommand{\rowstyle}[1]{\gdef\currentrowstyle{#1} #1\ignorespaces}
    
    %Images
    \graphicspath{ {images/} }                          %directory that your images are located in within your current directory
    
    %Diagrams
    \usepackage[latin1]{inputenc}
    \usepackage{tikz}
    \usepackage{tkz-berge}
    \usetikzlibrary{shapes,arrows}


%Bibliography
\usepackage[numbers,sort&compress]{natbib}   %for multiple references: sorts  (i.e. [1,2] NOT [2, 1] )
                                           				  %                                     compresses (i.e. [1-3] )
\usepackage[nottoc]{tocbibind}                            %add bibliography to table of contents


%Miscellaneous
\usepackage{dirtytalk}    %quotations: use \say  


%================== Header & Footer =========================
\usepackage{fancyhdr}
\usepackage{lastpage}      %ensures you can reference LastPage (i.e. Page 2 of 10)

\renewcommand{\headrulewidth}{0.4pt}		%Decorative Header line: thickness={0.4pt}
\renewcommand{\footrulewidth}{0.4pt}		%Decorative Footer line: thickness={0.4pt}

\setlength{\headheight}{13.6pt} 		%space b/w top of page & header
\setlength{\headsep}{0.3in}		%space b/w page header and body

%Make Header & Footer    
\pagestyle{fancy}
    \lhead{Stephanie Knill} 		% controls the left corner of the header
    \chead{} 					% controls the center of the header
    \rhead{} 					% controls the right corner of the header
    \lfoot{} 					% controls the left corner of the footer
    \cfoot{Page~\thepage\ of \pageref{LastPage}} 				% controls the center of the footer
    												%Page~\thepage\  if just want Page x
    \rfoot{}			 		% controls the right corner of the footer

% =============================== Document ===================================
\begin{document}

% Title Page
\title{MATH 302 --- Assignment 5 \\
\line(1,0){360} \\              %(slope x, y){length of line}
}
\author{
Stephanie Knill \\
54882113 \\
Due: February 22, 2016}

\date{}                   % Activate:  display a given date (e.g. {August 4} ) or no date (empty {} )
                                    %No activate: display current date
\maketitle

%\thispagestyle{empty}                   %Remove header from this (first) page. Change empty -> plain to keep numbering
%								-> Doesn't matter in this case (b/c title page)
%\cleardoublepage


% ================= Questions ================

\section*{Question 3: Section 6.1 \#8}

A royal family has children util it has a boy or until it has three children, whichever comes first. Assuming that each child is a boy with probability 1/2, we can construct a tree diagram for the the sample space $\Omega = \{B, GB, GGB, GGG\}$ (Figure \ref{tree}):

\begin{figure}[h]
\usetikzlibrary{positioning}

\tikzstyle{level 1}=[level distance=4cm, sibling distance=4cm]
\tikzstyle{level 2}=[level distance=3cm, sibling distance=3cm]
\tikzstyle{level 3}=[level distance=3cm, sibling distance=2cm]


\begin{tikzpicture}[level distance=5cm, grow'=right]
\tikzstyle{every node}=[]
    \node (Root) [] {Origin}
        child [] { node {B (1/2)}
            	     edge from parent
           	     node[above] {$\frac{1}{2}$}
        		}
        child { node {G}
                        child { node {GB (1/4)}
                                edge from parent
                                node[above] {$\frac{1}{2}$}
                        }
                        child { node {GG}
                    			child { node {GGB (1/8)}
    						edge from parent
    						node[above] {$\frac{1}{2}$}
    						}	
    					child { node {GGG (1/8)}
    						edge from parent
    						node[below] {$\frac{1}{2}$}
    						}
                                edge from parent
                                node[below] {$\frac{1}{2}$}
                        		}
            	edge from parent
            	node[below] {$\frac{1}{2}$}
         };
\end{tikzpicture}
\caption{Tree digram for a royal family}
\label{tree}
\end{figure}

\cleardoublepage
Let $X$ be the number of boys in the family. Then the expected value of the random variable $X$ is
\begin{align*}
	E[X] &= \sum_x x \cdot p(x) \\
	& = 1/2 \cdot (1) + 1/4 \cdot (1) + 1/8 \cdot (1) + 1/8 \cdot (0) \\
	& = \frac{7}{8}\\
	& = 0.875 
\end{align*}

Let $Y$ be the number of girls in the family. Then the expected value of the random variable $Y$ is
\begin{align*}
	E[Y] &= \sum_y y \cdot p(y) \\
	& = 1/2 \cdot (0) + 1/4 \cdot (1) + 1/8 \cdot (2) + 1/8 \cdot (3) \\
	& = \frac{7}{8}\\
	& = 0.875
\end{align*}

\cleardoublepage
\section*{Question 4: Section 6.2 \#2}

For a random variable $X$ that has distribution
$$p_X = 
\begin{pmatrix}					%\bmatrix for square braces
	0 & 1 & 2 & 4 \\
	1/3 & 1/3 & 1/6 & 1/6
\end{pmatrix}$$
we can find the expected value, variance, and standard deviation.

\textbf{Expected Value}
\begin{align*}
	E[X] & = \sum_x x \cdot p(x) \\
	& = 1/3\cdot(0) + 1/3 \cdot (1) + 1/6 \cdot (2) + 1/6 \cdot (4) \\
	& = \frac{4}{3} \\
	& \approx 1.33
\end{align*}

\textbf{Variance}
\begin{align*}
	Var[X] & = E[X^2]-E[X]^2
\end{align*}
Here,
\begin{align*}
	E[X^2] & = 1/3 \cdot (0)^2 + 1/3 \cdot (1)^2+ 1/6 \cdot (2)^2 + 1/6 \cdot (4)^2 \\
	& = 1/3 + 4/6 + 16/6 \\
	& = \frac{11}{3}
\end{align*}
Substituting this back in gives us
\begin{align*}
	Var[X] & = E[X^2]-E[X]^2 \\
	& = 11/3 - (4/3)^2 \\
	& = \frac{17}{9} \\
	& \approx 1.89
\end{align*}

\textbf{Standard Deviation}
\begin{align*}
	D[X] & = \sqrt{Var[X]} \\
	& = \sqrt{\frac{17}{9}} \\
	& \approx 1.37
\end{align*}

\section*{Question 5: Section 6.2 \#4}

Let $X$ be a random variable with expected value $E[X] = 100$ and variance $Var[X] = 15$. Then we can compute the following
\begin{enumerate}[(a)]
	\item $E(X^2)$
	\begin{align*}
		Var[X] & = E[X^2] - E[X]^2 \\
		E[X^2] & = Var[X] + E[X]^2\\
		& = 15 + 100^2 \\
		& = 10015
	\end{align*}
	
	\item $E[3X+10]$
	\begin{align*}
		E[3X+10] & = 3 \cdot E[X] + 10 \\
		& = 3 \cdot 100 + 10 \\
		& = 310
	\end{align*}
	
	\item $E[-X]$
	\begin{align*}
		E[-X] & = -1 \cdot E[X] \\
		& = -100 \\
	\end{align*}
	
	\item $Var[-X]$
	\begin{align*}
		Var[-X] & = (-1)^2 \cdot Var[X]\\
		& = Var[X] \\
		& = 15
	\end{align*}
	
	\item $D[-X]$
	\begin{align*}
		D[-X] & = \sqrt{Var[-X]} \\
		& = \sqrt{15} \\
		& \approx 3.87
	\end{align*}
\end{enumerate}






\end{document} 